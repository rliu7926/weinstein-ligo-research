\documentclass{article}
\usepackage[T1]{fontenc}
\usepackage[utf8]{inputenc}

\usepackage{cmbright}
\usepackage[T1]{fontenc}

\usepackage{multicol}

\usepackage{amsmath}
\usepackage{amsfonts}
\usepackage{amssymb}
\usepackage{tikz}
\usepackage{graphicx}
\graphicspath{  {./images/} }
\setlength{\parindent}{0pt}
\usepackage{changepage}
\usepackage{physics}
\usepackage{derivative}
\usepackage{bm}

\addtolength{\topmargin}{-.25in}	
\addtolength{\textheight}{.5in}
\addtolength{\oddsidemargin}{-.75in}
\addtolength{\textwidth}{1.5in}

\makeatletter
\newcommand*\bigcdot{\mathpalette\bigcdot@{.5}}
\newcommand*\bigcdot@[2]{\mathbin{\vcenter{\hbox{\scalebox{#2}{$\m@th#1\bullet$}}}}}
\makeatother

\DeclareMathOperator{\di}{d\!}
\newcommand*\Eval[3]{\left.#1\right\rvert_{#2}^{#3}}

\newcommand{\uvec}[1]{\boldsymbol{\hat{\textbf{#1}}}}
\newcommand{\vr}[1]{\textbf{#1}}

\newcommand{\thus}[0]{\; \; \longrightarrow \; \;}

\newcommand{\lag}{\mathcal{L}}
\newcommand{\ham}{\mathcal{H}}

\title{Gravitational Wave Research Overview}
\author{Ryan Liu}
\date{Last updated: \today}

\begin{document}


\maketitle

\begin{multicols*}{2}

\section{Research Proposal}

\subsection{Research Question}

\textbf{Question:} How far out in distance can we detect the gravitational wave emission of a black hole merger given a specific signal-to-noise ratio (SNR)? \\

\textbf{Initial Assumptions:}
\begin{itemize}
    \setlength\itemsep{0pt}
    \item The two black holes are of equal mass.
    \item The black holes are not spinning.
    \item The orientation of the gravitational wave detectors relative to the gravitational wave signal is random. 
\end{itemize}

\subsection{Objectives}

\begin{enumerate}
    \item Determine the redshift effect on gravitational waves on a cosmological scale. Create a function that generates waveforms based on mass and distance, factoring in redshift. 
    \item Create a function that generates random noise indistinguishable from O3 aLIGO data. 
    \item Create a function that inserts waveforms into aLIGO data, then determines the average SNR that is generated by the signal. 
    \item Create a function that determines the distance threshold for a particular SNR value and particular mass. Create a visualization for the results of this function. 
    \item Factor in any additional effects, such as glitches. 
\end{enumerate}

\section{Summary of Work}

\subsection{Cosmological Redshift}

\subsection{Random Noise Generator}

\subsection{SNR Calculation}

\section{Questions}








\end{multicols*}
\end{document}
