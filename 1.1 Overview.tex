\documentclass[10pt]{article}
\usepackage[T1]{fontenc}
\usepackage[utf8]{inputenc}

\usepackage{cmbright}
\usepackage[T1]{fontenc}

\usepackage{multicol}

\usepackage{amsmath}
\usepackage{amsfonts}
\usepackage{amssymb}
\usepackage{tikz}
\usepackage{graphicx}
\graphicspath{  {./images/} }
\setlength{\parindent}{0pt}
\usepackage{changepage}
\usepackage{physics}
\usepackage{derivative}
\usepackage{bm}

\addtolength{\topmargin}{-.5in}	
\addtolength{\textheight}{0.75in}
\addtolength{\oddsidemargin}{-.75in}
\addtolength{\textwidth}{1.5in}

\makeatletter
\newcommand*\bigcdot{\mathpalette\bigcdot@{.5}}
\newcommand*\bigcdot@[2]{\mathbin{\vcenter{\hbox{\scalebox{#2}{$\m@th#1\bullet$}}}}}
\makeatother

\DeclareMathOperator{\di}{d\!}
\newcommand*\Eval[3]{\left.#1\right\rvert_{#2}^{#3}}

\newcommand{\uvec}[1]{\boldsymbol{\hat{\textbf{#1}}}}
\newcommand{\vr}[1]{\textbf{#1}}

\newcommand{\thus}[0]{\; \; \longrightarrow \; \;}

\newcommand{\lag}{\mathcal{L}}
\newcommand{\ham}{\mathcal{H}}

\title{Gravitational Wave Research Overview}
\author{Ryan Liu}
\date{Last updated: \today}

\begin{document}


\maketitle

\begin{multicols*}{2}

\section{Research Proposal}

\subsection{Research Question}

\textbf{Question:} How far out in distance can we detect the gravitational wave emission of a black hole merger given a specific signal-to-noise ratio (SNR)? \\

\textbf{Initial Assumptions:}
\begin{itemize}
    \setlength\itemsep{0pt}
    \item The two black holes are of equal mass.
    \item The black holes are not spinning.
    \item The orientation of the detectors relative to the GW signal is random. 
\end{itemize}

\subsection{Initial Objectives}

\begin{enumerate}
    \item Determine the redshift effect on gravitational waves on a cosmological scale. Create a function that generates waveforms based on mass and distance, factoring in redshift. 
    \item Create a function that generates random noise indistinguishable from O3 aLIGO data. 
    \item Create a function that inserts waveforms into aLIGO data, then determines the average SNR that is generated by the signal. 
    \item Create a function that determines the distance threshold for a particular SNR value and particular mass. Create a visualization for the results of this function. 
    \item Factor in any additional effects. 
\end{enumerate}

\section{Summary of Progress: April}

\subsection{Cosmological Redshift}

A function was created to generate the waveform of a black hole merger given the masses of the two black holes and the distance of observation. The redshift effect was added based on Hubble's law. 

\subsection{Random Noise Generator}

To simulate the detection of a black hole merger by LIGO, a function was created to overlay the redshifted signal with noise based on various template PSDs. Then, the signal will be "found" with a particular waveform template at a particular SNR. 

\subsection{SNR Calculation}

To answer the research question, the functions from the previous two sections were used to look at the impact of distance of observation, imprecision of the template waveform, LIGO sensitivity, black hole mass, and random noise on the SNR of detection. 

\section{Summary of Progress: May}

\subsection{Recalculated Redshift}

Because a linear approximation of redshift is only accurate at small distances, a more refined redshift approximation as a function of $\Omega_m$ and $\Omega_v$ was calculated. These values were checked with the AstroPy package, using the Planck 2018 values for cosmological constants. 

\subsection{SNR Expectation Value Function}

In order to remove the effect of noise on determining the SNR of a BH signal, the function created in Section (2.2) was modified to remove noise. Additionally, to speed up computation the calculations were moved into the frequency spectrum. 

\subsection{SNR Expectation Simulations}

Similar to Section (2.3), the SNR was calculated for black hole mergers of varying masses, distances, sensitivity curves, and templates. The maximum distance that a black hole merger can be observed as a function of frequency was also determined. 

\subsection{SNR Expectatation vs. Noise}

The results of SNR calculations using the functions from Section (2.2) and (3.2) were compared to determine whether they agree with one another. Surprisingly, the expectation value appears to be consistently higher than the median SNR when noise is added for the same BH merger signal. 

\subsection{Distance Estimation Function}

Because the functions created in (2.2) and (3.2) determine SNR as a function of multiple parameters (including distance) and cannot be inverted, in order to answer the original research question a function was created using the concept of parameter estimation to determine distance as a function of the desired SNR threshold. 

\subsection{Distance Estimation Simulations}

The maximum distance a black hole merger can be detected as a function of mass, LIGO sensitivity, and threshold SNR was calculated in this section. Due to inaccuracies in function convergence, a more detailed look into to results of (3.3) was also conducted. 

\subsection{Spin Simulations}

The effect of spin on GW signals and their relative SNR were investigated. 





\section{Questions}

\begin{itemize}
    \item Why is there a deviation between the expectation SNR and the median of multiple runs of the simulated SNR (with noise)? 
    \item Is there any better way to determine the maximum redshift/distance?
    \item Why can't the spin parameter for the $x$ and $y$ axes be set to anything but 0? Is it necessary/important to investigate those spins? 
    \item What next steps should I take? 
\end{itemize}






\end{multicols*}
\end{document}
