\documentclass{article}
\usepackage[T1]{fontenc}
\usepackage[utf8]{inputenc}

\usepackage{cmbright}
\usepackage[T1]{fontenc}

\usepackage{multicol}

\usepackage{amsmath}
\usepackage{amsfonts}
\usepackage{amssymb}
\usepackage{tikz}
\usepackage{graphicx}
\graphicspath{  {./images/} }
\setlength{\parindent}{0pt}
\usepackage{changepage}
\usepackage{physics}
\usepackage{derivative}
\usepackage{bm}
\usepackage[colorlinks=true, linkcolor=blue, urlcolor=blue, citecolor=blue, anchorcolor=blue]{hyperref}

\addtolength{\oddsidemargin}{-.25in}
\addtolength{\textwidth}{0.5in}

\makeatletter
\newcommand*\bigcdot{\mathpalette\bigcdot@{.5}}
\newcommand*\bigcdot@[2]{\mathbin{\vcenter{\hbox{\scalebox{#2}{$\m@th#1\bullet$}}}}}
\makeatother

\DeclareMathOperator{\di}{d\!}
\newcommand*\Eval[3]{\left.#1\right\rvert_{#2}^{#3}}

\newcommand{\uvec}[1]{\boldsymbol{\hat{\textbf{#1}}}}
\newcommand{\vr}[1]{\textbf{#1}}

\newcommand{\thus}[0]{\; \; \longrightarrow \; \;}

\newcommand{\lag}{\mathcal{L}}
\newcommand{\ham}{\mathcal{H}}

\title{Advanced LIGO Random Noise Generation}
\author{Ryan Liu}
\date{Last updated: \today}

\begin{document}

\maketitle

\section{Resources Used}

\begin{enumerate}
    \item \textit{Introduction to Bayesian Parameter Estimation for Compact Binary Coalescences} by Sylvia Biscoveanu (GW Open Data Workshop #3 presentation)
    \item B P Abbott \textit{et al} 2020 \textit{Class. Quantum Grav.} \textbf{37} 055002
\end{enumerate}


\section{Notes}

The idea behind parameter estimation is \textbf{Bayes' Theorem}:
\begin{equation}
    P (A | B) = \frac{P (B | A) \cdot P (A)}{P(B)}
\end{equation}
which we are familiar with from elementary statistics. The GW detection version of this equation is 
\begin{equation}
    p(\theta | d,H) = \frac{p(d,\theta, H) p(\theta | H)}{p(d|H)} = \frac{\lag(d | \theta, H) \pi(\theta | H)}{\mathcal{Z}_H}
\end{equation}
where $\theta$ denotes the parameters of interest, $d$ is the data collected by LIGO, and $H$ is the particular gravitational wave model. Breaking down into components: 
\begin{itemize}
    \setlength{\itemsep}{0pt}
    \item $p(\theta |d,H)$: posterior 
    \item $p(d|\theta, H)$: likelihood (of getting data $d$ given parameters $\theta$ and model $H$)
    \item $p(\theta | H)$: prior (initial guess/probability of parameters $\theta$ under model $H$)
    \item $p(d | H)$: evidence (normalization factor)
\end{itemize}
The Fourier transform of the data (frequency domain) is given by 
\begin{equation}
    \tilde{d}(f) = \tilde{n}(f) + \tilde{h}(\theta; f)
\end{equation}
where $\tilde{n}(f)$ is the noise (which we hope to generate here). We assume that \textit{the noise is stationary and Gaussian}, although this is not particularly true due to the presence of glitches. This means that the strain/data follows a unit Gaussian distribution about $\sqrt{\text{PSD}}$ when there is no signal; i.e. the histogram of the value of strain divided by $\sqrt{\text{PSD}}$ will produce a normal curve. \\

\section{Modeling Random aLIGO Noise}

An outline of the process in developing the function follows: 
\begin{enumerate}
    \item Using the redshifted waveform creator function from Section 2.1, create the waveform that will be detected. 
    \item Generate the frequency series by using the "pycbc.psd.from\_string" function:
    \begin{enumerate}
        \setlength{\itemsep}{0pt}
        \item psd\_name: the name of the preloaded sensitivity / PSD curve
        \item length: number of "points" needed to construct the PSD, equal to 
        \begin{equation}
            \frac{\text{number of simulations}}{\text{resolution}} + 1
        \end{equation}
        where the resolution is usually $1/16$ and the number of simulations $4096$
        \item delta\_f: the resolution, i.e. difference in frequency between each "point"
        \item low\_freq\_cutoff: the lowest frequency that noise will be generated from
    \end{enumerate}
    \item Turn the frequency-series PSD into a time-series of Gaussian noise using the pycbc.noise.noise\_from\_psd function:
    \begin{enumerate}
        \setlength{\itemsep}{0pt}
        \item length: number of "points" of data, equal to 
        \begin{equation}
            \text{seconds of noise} \times \text{resolution}
        \end{equation}
        \item delta\_t: resolution, i.e. time interval between each data point
        \item psd: the PSD generated in the previous step
        \item seed: a random number
    \end{enumerate}
    \item Resize the GW signal created in step 1 to the length of the random noise sample created in step 3
    \item Add the GW signal to the random noise sample
    \item Create and resize the waveform template that will be used to detct the signal in the same manner as steps 1 and 3, making sure that no redshifting occurs
    \item Calculate the SNR time series by using the matched\_filter function
    \item Report the maximum SNR, and check that it occurs at the same time stamp as where the waveform was inserted
\end{enumerate}


















\end{document}